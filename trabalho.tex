\documentclass[12pt,a4paper]{article}
\usepackage[utf8]{inputenc}
\usepackage[portuguese]{babel}
\usepackage[T1]{fontenc}
\usepackage{graphicx}
\usepackage[left=2cm,right=2cm,top=2cm,bottom=2cm]{geometry}
\usepackage{indentfirst}




\begin{document}
\thispagestyle{empty}
\begin{center}
{\Large\sc Universidade Federal de Minas Gerais}

{\large Sistemas Digitais}

Prof. Julio Conway

\vfill
{\LARGE\bf Sistema de Destravamento Presencial Sequencial}

\medskip
Cleber Vargas Borges

Gabriel Machado de Castro Fonseca

\vfill
Belo Horizonte - MG

2015
\end{center}
\newpage

\section{Introdução}

A eletrônica talvez seja considerada o maior avanço tecnológico do século XX. A sua evolução se deu de forma exponencial, sendo hoje em dia uma tecnologia acessível e praticamente onipresente. Enquanto no início lidávamos com componentes grandes, delicados e pouco eficientes, com o passar das décadas, foram-se criando alternativas menores, menos dispendiosas e muito mais rápidas.

Com o uso da eletrônica digital e as possibilidades surgidas com a miniaturização dos circuitos, foi possível criar máquinas programáveis que cabem em nossos bolsos e executam milhões de cálculos a mais por segundo do que os primeiros computadores a válvulas. A lógica digital permite uma extrema maleabilidade na solução de um problema, fazendo com que blocos construtivos simples componham um circuito complexo, projetado para lidar com uma situação específica.

O objetivo deste trabalho é o de demonstrar a modelagem de um circuito lógico seqüencial que funcione como um sistema de destravamento de um cofre. São necessárias 3 pessoas para se abrir o cofre, cada uma deve digitar seu código de acesso na ordem certa para que a porta se destrave. Ao identificar corretamente a primeira pessoa uma luz vermelha se acende, a segunda pessoa a ser identificada faz com que se acenda uma luz amarela e, finalmente, quando a terceira pessoa é identificada acende-se uma luz verde e a tranca é aberta por 4 segundos, retornando então ao estado inicial. Qualquer erro de identificação ou na ordem de entrada das senhas faz com que se retorne ao estado inicial.

\section{Definição das Senhas}

Definimos que as senhas seriam, conforme instruções do trabalho, G0, C7 e J2. Conforme tabela ASCII, a representação das senhas, considerando maiúscula e minúscula, seria

\bigskip

\begin{table}[hb]
\begin{center}
\begin{tabular}{c||cccccccc|cccc}
Senha & $I_0$ & $I_1$ & $I_2$ & $I_3$ & $I_4$ & $I_5$ & $I_6$ & $I_7$ & $I_8$ & $I_9$ & $I_{10}$ & $I_{11}$ \\
\hline
G0 & {\bf 0} & {\bf 1} & {\bf 0} & 0 & 0 & 1 & 1 & 1 & {\bf 0} & 0 & 0 & 0 \\
g0 & {\bf 0} & {\bf 1} & {\bf 1} & 0 & 0 & 1 & 1 & 1 & {\bf 0} & 0 & 0 & 0 \\
\hline
\hline
C7 & {\bf 0} & {\bf 1} & {\bf 0} & 0 & 0 & 0 & 1 & 1 & {\bf 0} & 1 & 1 & 1 \\
c7 & {\bf 0} & {\bf 1} & {\bf 1} & 0 & 0 & 0 & 1 & 1 & {\bf 0} & 1 & 1 & 1 \\
\hline
\hline
J2 & {\bf 0} & {\bf 1} & {\bf 0} & 0 & 1 & 0 & 1 & 0 & {\bf 0} & 0 & 1 & 0 \\
j2 & {\bf 0} & {\bf 1} & {\bf 1} & 0 & 1 & 0 & 1 & 0 & {\bf 0} & 0 & 1 & 0 \\
\end{tabular}
\end{center}
\caption{Codificação ASCII das senhas do sistema}
\end{table}

Pode-se perceber que os números em destaque são redundantes para a identificação das senhas, uma vez que se repetem ou seguem um padrão óbvio, variando apenas um bit entre o caractere maiúsculo e minúsculo. Dessa forma podemos adotar apenas 6 bits para a letra e 3 bits para os números, uma vez que nenhuma senha possuí número maior que 8.
\end{document}